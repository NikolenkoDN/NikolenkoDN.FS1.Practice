\documentclass[12pt,a4paper]{article}
\usepackage[ utf8 ]{inputenc}
\usepackage[english,russian]{babel}
\usepackage[russian]{babel} 
\usepackage[T2A]{fontenc} 
\usepackage{amsmath,amssymb}
\usepackage{pbox}
\usepackage{indentfirst}
\usepackage{misccorr}
\usepackage{titlesec}
\usepackage{graphicx}
\pagestyle{empty} % нумерация выкл.
\usepackage[x11names]{xcolor}
\definecolor {brightmaroon} {rgb} {0,76, 0,13, 0,28}
\definecolor {royalazure} {rgb} {0,0, 0,22, 0,66}
\usepackage[colorlinks = true, linkcolor = royalazure]{ hyperref }
\usepackage{ tikz, tkz-fct, pgfplots }
\usepackage[a4paper, total={170 mm, 257 mm},left=20mm, top=20mm ]{ geometry }
% ----------------- Команды -----------------
\newcommand {\eps } { \varepsilon }
\newcommand \tline[2]{ $ \underset { \text {# 1}} { \text { \underline { \hspace {# 2}}}} $ }

% ----------------- Установить путь к графике -----------------
\graphicspath{{img/}}

\begin{document}
	\pagestyle {пусто}
	\centerline { \large Министерство науки и высшего образования}	
	\centerline { \large Федеральное государственное бюджетное образовательное}
	\centerline { \large учреждение высшего образования}
	\centerline { \large  `` Московский государственный технический университет}
	\centerline { \large имени Н.Э. Баумана}
	\centerline { \large (национальный исследовательский университет) ''}
	\centerline { \large (МГТУ им. Н.Э. Баумана)}
	\hrule
	
	\vspace{0.5 cm}
	
	\begin{figure}[h]
		\center
		\includegraphics[height=0.35\linewidth]{bmstu-logo-small}
	\end{figure} 
	
	\begin{center}
		\large
		\begin{tabular} {c}
			Факультет ``Фундаментальные науки'' \\
			Кафедра ``Высшая математика''		
			\end {tabular}
		\end{center}
		
		\vspace{0.5 cm}
		
		\begin{center}
			\LARGE \bf	
			\begin{tabular} {c}
				\textc{Отчёт} \\
				по учебной практике \\
				за 3 семестр 2020 --- 2021 гг.
				\end {tabular}
			\end{center}
			
			\vspace{0,5}
			
			\begin{center}
				\large
				\begin{tabular}{p{5.3cm} 11}
					\pbox{5.45cm} {
						Руководитель практики, \\
						ст. преп. кафедры ФН1} & $\underset{\text{(подпись)}}{\underline{\hspace{5cm\textwidth}}}$ & Кравченко О.В. \\ [0.5cm]
					студент группы ФН1--31 & $\underset{\text{(подпись)}}{\underline{\hspace{5cm\textwidth}}}$ & Николенко Д.Н.
				\end{tabular}
			\end{center} 
			
			\vfill
			
			\begin{center}
				\large
				\begin{tabular} {c}
					Москва, \\
					2020 г.
				\end{tabular}
			\end{center}
			
			
			\newpage	
			\tableofcontents
			
			\newpage
			\section{Цели и задачи практики}	
			\subsection{Цели}
			--- развитие компетенций, способствующих успешному освоению материала бакалавриата и необходимых в будущей профессиональной деятельности.
			
			\subsection{Задачи}
			\begin{enumerate}
				\item Знакомство с теорией рядов Фурье, и теорией интегральный уравнений.
				\item Развитие умения поиска необходимой информации в специальной литературе и других источниках.
				\item Развитие навыков составления отчётов и презентации результатов.
			\end{enumerate}
			
			\subsection{Индивидуальное задание}	
			\begin{enumerate}
				\item Изучить способы отображения математической информации в системе вёртски \LaTeX.
				\item Изучить возможности  системы контроля версий \textsf{Git}.
				\item Научиться верстать математические тексты, содержащие формулы и графики в системе \LaTeX.
				Для этого, выполнить установку свободно распространяемого дистрибутива \textsf{TeXLive} и оболочки \textsf{TeXStudio}.
				\item Оформить в системе \LaTeX типовые расчёты по курсу математического анализа согласно своему варианту.
				\item Создать аккаунт на онлайн ресурсе \textsf{GitHub} и загрузить исходные \textsf{tex}--файлы 
				и результат компиляции в формате \textsf{pdf}.
				\item Решить индивидуальное домашнее задание согласно своему варианту, и оформить решение с учётов пп. 1---4.
			\end{enumerate} 
			
			\newpage
			\section{Отчёт}
			Интегральные уравнения имеют большое прикладное значение, являясь мощным
			орудием исследования многих задач естествознания и техники: они широко используются
			в механике, астрономии, физике, во многих задачах химии и биологии. Теория линейных
			интегральных уравнений представляет собой важный раздел современной математики,
			имеющий широкие приложения в теории дифференциальных уравнений, математической
			физике, в задачах естествознания и техники. Отсюда владение методами теории
			дифференциальных и интегральных уравнений необходимо приклажному математику, при решении задач
			механики и физики.
			
			\newpage
			\section{Индивидуальное задание}
			\subsection{Ряды Фурье и интегральное уравнение Вольтерры.}
			
			\subsubsection*{\center Задача № 1.}
			{\bf Условие.~}
			Разложить в ряд Фурье заданную функцию $f(x)$, построить графики $f(x)$ и суммы ее ряда Фурье. Если не указывается, какой вид разложения в ряд необходимо представить, то требуется разложить функцию либо в общий тригонометрический ряд Фурье, либо следует выбрать оптимальный вид разложения в зависимости от данной функции.
			% ---------------------------- Problem 1----------------------------------
			\[
			\begin{equation}
				f(x) = ch(ax), 	& -\pi \leqslant x \leqslant \pi
			\end{equation}
			\]
			{\bf Решение.~}	
			%График
			\begin{center}
				\begin{tikzpicture}
					\begin{axis}[xmin=-4,	xmax=4, 	ymin=-1,	ymax=3,
						width=0.5\textwidth,
						height=0.4\textwidth,
						axis x line=middle,
						axis y line=middle, 
						every axis x label/.style={at={(current axis.right of origin)},anchor=west},
						every inner x axis line/.append style={|-latex'},
						every inner y axis line/.append style={|-latex'},
						minor tick num=1,			
						axis equal=true,
						xlabel=$x$, 
						ylabel=$y$,          
						samples=100,
						clip=true,
						]
						\addplot[color=black, line width=1.5pt,domain=-3.14:3.14]{cosh(x)};
					\end{axis}
				\end{tikzpicture}
			\end{center}
			\noindent
			Построим тригонометрический ряд Фурье вида
			$$
			f(x)=\frac{a_0}{2}+\sum_{n=1}^\infty 
			\left( a_n\cos{(nx)}\right).
			$$
			\noindent
			Вычислим коэффициенты
			$$
			\begin{array}{rcl}
				a_0 &=& \displaystyle\frac{2}{\pi}\left
				\left( 
				\int\limits_0^\pi
				ch(ax)\,dx \right)
				= \frac{2sh(a\pi)}{a\pi},												\\[12pt]
				a_n &=& \displaystyle\frac{2}{\pi}\left(
				\int\limits_0^\pi
				\ch(ax) \cdot cos(nx)\,dx \right) ={}									\\[12pt]
				&=& \displaystyle\frac{2}{\pi}\left(
				\frac{e^{-a\pi}((e^{2a\pi} + 1)n sin(n\pi) + (ae^{2a\pi} -a)cos(n\pi))}{2(n^2 + a^2)}\right) = 	\\[12pt]
				&=& \displaystyle\frac{e^{-a\pi}(ae^{2a\pi} -a)cos(n\pi)}{2(n^2 + a^2)} = \displaystyle\frac{e^{-a\pi}(ae^{2a\pi} -a)(-1)^n}{2(n^2 + a^2)} .	\\[12pt]
			\end{array}
			$$
			Применив теорему Дирихле о поточечной сходимости ряда Фурье, видим, что построенный ряд Фурье сходится 
			к периодическому (с периодом $T=2\pi$) продолжению исходной функции. Рисунок соответствует случаю, когда $a>0$. При этом значения функции на концах промежутка $[-\pi, \pi]$ совпадают: $f(-\pi+0) = \displaystyle\frac{e^{-a\pi}+e^{a\pi}}{2}$, $f(\pi+0) = \displaystyle\frac{e^{a\pi}+e^{-a\pi}}{2}$. $S(-\pi)=S(\pi)=\displaystyle\frac{1+e^{2a\pi}}{2e^{a\pi}}$.
			График функции $S(x)$ имеет следующий вид
			\begin{center}
				\begin{tikzpicture}
					\begin{axis}[xmin=-11, xmax=11, ymin=-1, ymax=12.6,
						width=0.8\textwidth,
						height=0.4\textwidth,
						axis x line=middle,
						axis y line=middle, 
						every axis x label/.style={at={(current axis.right of origin)},anchor=west},
						every inner x axis line/.append style={|-latex'},
						every inner y axis line/.append style={|-latex'},
						minor tick num=1,			
						axis equal=true,
						xlabel=$x$, 
						ylabel=$S(x)$,          
						samples=100,
						clip=true,
						]
						\addplot[color=black, line width=1.5pt,domain=-9.42:-3.14]{cosh(x+6.28)};
						\addplot[color=black, line width=1.5pt,domain=-3.14:3.14]{cosh(x)};
						\addplot[color=black, line width=1.5pt,domain=3.14:9.42]{cosh(x-6.28)};
						\addplot[color=black, line width=1.5pt,domain=-15.7:-9.42] {cosh(x+12.56)};
						\addplot[color=black, line width=1.5pt,domain=9.42:15.7]{cosh(x-12.56)};
						\addplot[thick,dashed] coordinates {(-3.14, 0) (-3.14, 11.592)};
						\addplot[thick,dashed] coordinates {(3.14, 0) (3.14, 11.592)};
						\addplot[thick,dashed] coordinates {(9.42, 0) (9.42, 11.592)};
						\addplot[thick,dashed] coordinates {(-9.42, 0) (-9.424, 11.592)};
						\addplot[
						mark=*,
						mark options={fill=black, draw=black},
						only marks,
						] coordinates {(-9.42, 11.592) (-3.14, 11.592) (3.14, 11.592) (9.42, 11.592)};
					\end{axis}
				\end{tikzpicture}
			\end{center}
			\noindent
			\textbf{Ответ:}
			\[
			\begin{split}
				&f(x)=\frac{2sh(a\pi)}{a\pi}+\sum_{n=1}^\infty\left[\displaystyle\frac{(e^{-a\pi}(ae^{2a\pi} -a)(-1)^n}{2(n^2 + a^2)}cos(nx)\right]; \\
				&S(n)=\frac{1+e^{2a\pi}}{2e^{a\pi}}, \text{ при } n\in\mathbb{Z}.
			\end{split}
			\]
			
			% ---------------------------- Problem 2----------------------------------
			\subsubsection*{\center Задача № 2.}
			{\bf Условие.~}
			Для заданной графически функции $y(x)$ построить ряд Фурье в комплексной форме, изобразить график суммы построенного ряда
			
			%График
			\begin{center}
				\begin{tikzpicture}
					\begin{axis}[xmin=-1,	xmax=5, 	ymin=-1,	ymax=3,
						width=0.5\textwidth,
						height=0.4\textwidth,
						axis x line=middle,
						axis y line=middle, 
						every axis x label/.style={at={(current axis.right of origin)},anchor=west},
						every inner x axis line/.append style={|-latex'},
						every inner y axis line/.append style={|-latex'},
						minor tick num=1,			
						axis equal=true,
						xlabel=$x$, 
						ylabel=$y$,          
						samples=100,
						clip=true,
						]
						\addplot[color=black, line width=1.5pt,domain=0:2] {-\x+2};
						\addplot[color=black, line width=1.5pt,domain=2:4]{0};
						%	\addplot[thick,dashed] coordinates {(2,0) (2,-1)};
						\addplot[
						mark=*,
						mark options={fill=black, draw=black},
						only marks,
						] coordinates {(2, 0)};
					\end{axis}
				\end{tikzpicture}
			\end{center}
			
			\noindent
			\textbf{Решение.}\\
			
			\noindent
			Ряд Фурье в комплексной форме имеет следующий вид
			\[
			f(x) = \sum_{n=-\infty}^\infty c_n e^{i\omega nx},\quad c_n=\frac{1}{T}\int\limits_a^b f(x) e^{-i\omega nx}dx,\,\omega=\frac{2\pi}{T}.
			\]
			В нашем примере $ a=0,b=4,T=4,\omega=\pi/2$, 
			найдем коэффицинеты $c_n,\,n=0,\pm1,\pm2,\ldots$
			где $\omega=2\pi/T,\,T=4.$
			$$
			\begin{array}{rcl}
				c_0 &=&\displaystyle\frac{1}{4} \int\limits_0^4 f(x)dx=\frac{1}{4}\left(\int\limits_0^2 (-x+2)dx\right)=-\frac{1}{2},\\[12pt]
				c_n &=&\displaystyle\frac{1}{4}\left(
				\int\limits_0^2
				(-x+2)e^{-i\frac{\pi}{2}nx}dx\right) ={}\\[12pt]
				&=&\displaystyle\frac{isin(n\pi)-cos(n\pi)-in\pi +1}{\pi^2n^2} = \\[12pt]
				&=&\displaystyle\frac{1-cos(n\pi)-in\pi}{\pi^2 n^2} = \frac{1-(-1)^n -in\pi}{\pi^2 n^2}.
			\end{array}
			$$
			\noindent
			Применив теорему Дирихле о поточечной сходимости ряда Фурье, видим, что построенный ряд Фурье сходится 
			к периодическому (с периодом $T=4$) продолжению исходной функции. График функции $S(x)$ имеет вид
			\begin{center}
				\begin{tikzpicture}
					\begin{axis}[xmin=-8, xmax=8, ymin=-1, ymax=3,
						width=0.8\textwidth,
						height=0.4\textwidth,
						axis x line=middle,
						axis y line=middle, 
						every axis x label/.style={at={(current axis.right of origin)},anchor=west},
						every inner x axis line/.append style={|-latex'},
						every inner y axis line/.append style={|-latex'},
						minor tick num=1,			
						axis equal=true,
						xlabel=$x$, 
						ylabel=$S(x)$,          
						samples=100,
						clip=true,
						]
						\addplot[color=black, line width=1.5pt,domain=-8:-6] {-\x-6};
						\addplot[color=black, line width=1.5pt,domain=-6:-4]{0};
						\addplot[color=black, line width=1.5pt,domain=-4:-2] {-\x-2};
						\addplot[color=black, line width=1.5pt,domain=-2:0]{0};
						\addplot[color=black, line width=1.5pt,domain=0:2] {-\x+2};
						\addplot[color=black, line width=1.5pt,domain=2:4]{0};
						\addplot[color=black, line width=1.5pt,domain=4:6] {-\x+6};
						\addplot[color=black, line width=1.5pt,domain=6:8]{0};
						\addplot[
						mark=*,
						mark options={fill=black, draw=black},
						only marks,
						] coordinates {(-8, 1) (-4, 1) (0, 1) (4, 1) (8, 1)};
					\end{axis}
				\end{tikzpicture}
			\end{center}
			
			\noindent
			\textbf{Ответ:}
			\[
			\begin{split}
				&f(x)=\sum_{n=-\infty}^\infty\left[ \frac{1-(-1)^n -in\pi}{\pi^2 n^2}\right] e^{\tfrac{i\pi nx}{2}},~ x\ne 2n; \\
				&S(2n)=1,\quad\text{при}~n\in\mathbb{Z}.
			\end{split}
			\]
			
			% ---------------------------- Problem 3----------------------------------
			\subsubsection*{\center Задача № 3.}
			{\bf Условие.~}\\
			Найти резольвенту для интегрального уравнения Вольтерры со следующим ядром 
			$$K(x,t)=(t-x)e^{x^4-t^4}.$$
			
			\noindent
			{\bf Решение.~}\\
			\noindent
			Запишем интегральное уравнение Вольтерры
			$$
			y(x)= f(x) +\int\limits_0^x (t-x)e^{x^4-t^4} y(t)dt.
			$$
			Из рекурентных соотношений получаем
			$$
			\begin{array}{rcl}
				K_1(x,t)&=&\displaystyle (t-x)e^{x^4-t^4}, \\[12pt]
				K_2(x,t)&=&\displaystyle\int\limits_t^x K(x,s)K_1(s,t)ds = \int\limits_t^x (s-x)e^{x^4-s^4} (t-s)e^{s^4-t^4} ds = 
				\displaystyle-\frac{(t-x)^3}{6}\cdot e^{x^4-t^4},\\[12pt]
				K_3(x,t)&=&\displaystyle\int\limits_t^x K(x,s)K_2(s,t)ds = -\int\limits_t^x (s-x)e^{x^4-s^4} \cdot \frac{(t-s)^3}{6} e^{s^4-t^4} ds = \frac{(t-x)^5\cdot e^{x^4-t^4}}{120}.\\[12pt]
				K_j(x,t)&=&\displaystyle\frac{(-1)^{j-1}\cdot e^{x^4-t^4}\cdot (t-x)^{2j-1}}{(2j-1)!}\!\!\!\!\!\!\!\!\!\!\!\!\! \quad \quad \quad, \quad j=\mathbb{N}.
			\end{array}
			$$
			Подставляя это выражение для итерированных ядер, найдем резольвенту
			$$ 
			R(x,t,\lambda)=-e^{x^4-t^4}\cdot \sum_{j=1}^\infty \frac{(-1)^{j-1}\cdot(t-x)^{2j-1}}{(2j-1)!}\!\!\!\!\!\!\!\!\!\!
			\quad\quad,\quad j=1,2,\ldots, \lambda = 1, 
			$$
			
			
			\newpage
			\addcontentsline{toc}{section}{ Список литературы ~}
			\begin{thebibliography}{99}
				\bibitem{book01} Львовский С.М. Набор и вёрстка в системе \LaTeX,\,2003.
				\bibitem{book02} Краснов М.Л., Киселев А.И., Макаренко Г.И. Интегральные уравнения. М.:~Наука,\,1976.
				\bibitem{book03} Васильева А. Б., Тихонов Н. А. Интегральные уравнения. --- 2-е изд., стереотип. --- М:~ФИЗМАТЛИТ,\,2002.
			\end{thebibliography}
			
		\end{document}